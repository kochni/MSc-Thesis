
\documentclass[11pt,a4paper,twoside,openright]{report}
\usepackage[english]{ETHDAsfs}

\usepackage{pdfpages} %%to include the confirmation of originality (plagiarism
\usepackage{amsbsy} %% for \boldsymbol and \pmb{.}
\usepackage{amssymb} %% calls  amsfonts...
\usepackage{graphicx} %-- für PostScript-Grafiken (besser als  psfig!)
\usepackage{algorithm, algpseudocode}
%\usepackage[draft]{graphicx} % grafics shown as boxes --> faster compilation
%
\usepackage[longnamesfirst]{natbib}%was {sfsbib}%- Für  Literatur-Referenzen
%           ^^^^^^^^^^^^^^ 1) "Hampel, Ronchetti, ..,"  2) "Hampel et al"
% Engineers (and other funny people) want to see [1], [2] 
% ---> use 'numbers' : \usepackage[longnamesfirst,number]{natbib}
%
%
\usepackage{texab}%- 'tex Abkürzungen' /u/sfs/tex/tex/latex/texab.sty
        %%- z.B.  \R, \Z, \Q, \Nat für reelle, ganze, rationale, natürl. Zahlen;
        %%-       \N   (Normalvert.)  \W == Wahrscheinlichkeit .....
        %%-  \med, \var, \Cov, \....
        %%-  \abs{x} == |x|   und   \norm{y} ==  || y ||   (aber anständig)
%% NOTE: texab contains many useful definitions and "shortcuts". It is
%% worth to open the file and have a look at them. HOWEVER, some
%% definitions are a bit can lead to conflicts with other packages. You
%% might for example want to comment out the line defininf \IF as an
%% operator when working with the algorithmic package, or to comment out
%% the line defining a command \Cite with working with the Biblatex package  
\usepackage{amsmath}
%\usepackage{mathrsfs}% Raph Smith's Formal Script font --> provides \mathscr
\usepackage[utf8]{inputenc}% <<------- Unicode, *NOT* iso-latin1 !
\usepackage{ae}% A[lmost] E[uropean] Fonts
\usepackage{enumerate}% Fuer selbstdefinierte Nummerierungen
%--------
\usepackage{relsize}%-> \smaller (etc) used here
\usepackage{color} %% to allow coloring in code listings
\usepackage{listings}% Fuer R-code, C-code, ....  and settings for these:
\definecolor{Mygrey}{gray}{0.75}% for linenumbers only!
\definecolor{Cgrey}{gray}{0.4}% for comments
\lstloadlanguages{R}
%%--- first version of "listings of R"-style : ---------------------------
% %% using \smaller here: makes R code listings use a *small* font:
% \lstset{language=R,basicstyle=\smaller[2],commentstyle=\rmfamily\smaller,
%   showstringspaces=false,xleftmargin=4ex,
%   literate={<-}{{$\leftarrow$}}1 {~}{{$\sim$}}1}
% \lstset{escapeinside={(*}{*)}} % for (*\ref{ }*) inside lstlistings (Scode) 
%\newcommand{\lil}[1]{\lstinline|#1|}
%%--- newer version of "listings of R"-style : ---------------------------
\lstset{%% Help, e.g. --> https://en.wikibooks.org/wiki/LaTeX/Source_Code_Listings
language=R,
basicstyle=\ttfamily\scriptsize,%%- \small > \footnotesize > \scriptsize > \tiny
%commentstyle=\ttfamily\color{Cgrey},
commentstyle=\itshape\color{Cgrey},
numbers=left,
numberstyle=\ttfamily\color{Mygrey}\tiny,
stepnumber=1,
numbersep=5pt,
backgroundcolor=\color{white},
showspaces=false,
showstringspaces=false,
showtabs=false,
frame=single,
tabsize=2,
captionpos=b,
breaklines=true,
%breakatwhitespace=false,
keywordstyle={},
morekeywords={},
xleftmargin=4ex, 
literate={<-}{{$\leftarrow$}}1 {~}{{$\sim$}}1}
\lstset{escapeinside={(*}{*)}} % for (*\ref{ }*) inside lstlistings (Scode) 
%%----------------------------------------------------------------------------

%%------- Theorems ---
\newtheorem{definition}{Definition}[subsection]
\newtheorem{lemma}[definition]{Lemma}
\newtheorem{theorem}[definition]{Theorem}
\newtheorem{Coro}[definition]{Corollary}
\theoremstyle{definition} 
\newtheorem{example}[definition]{Example}
\newtheorem*{note}{Note}
\newtheorem*{remark}{Remark}

% for algorithmic
\newcommand{\Input}{\textbf{Input:}}
\newcommand{\Output}{\textbf{Output:}}

% \def\MR#1{\href{http://www.ams.org/mathscinet-getitem?mr=#1}{MR#1}}

% \newcommand{\Lecture}[3]{\marginpar{#3.#2.#1}}
% \newcommand{\Fu}{\mathcal{F}}
\newcommand{\aatop}[2]{\genfrac{}{}{0pt}{}{#1}{#2}}

%\renewcommand{\theequation}{\arabic{equation}}
\numberwithin{equation}{subsection}

%%%%%%%%%%%%%%%%%%%%%%%%%%%%%%%%%%%%%%%%%%%%%%%%%
%%% Path for your figures                      %%%
%%%%%%%%%%%%%%%%%%%%%%%%%%%%%%%%%%%%%%%%%%%%%%%%%
% Set the paths where all figures are taken from:
\graphicspath{{Plots/}}

%%%%%%%%%%%%%%%%%%%%%%%%%%%%%%%%%%%%%%%%%%%%%%%%%
%%% Define your own commands here             %%%
%%%%%%%%%%%%%%%%%%%%%%%%%%%%%%%%%%%%%%%%%%%%%%%%%
\newcommand{\Bruch}[2]{{}^{#1}\!\!/\!_{#2}}
\renewcommand{\labelenumi}{\roman{enumi}.)}


\begin{document}
\bibliographystyle{chicago}% ---> Hampel,F., E.Ronchetti,... W.Stahel(1986) ...
 %was \bibliographystyle{sfsbib}\citationstyle{dcu} %OR DEFAULT : \citationstyle{agsm}

\pagenumbering{roman}%- roman numbering for first few pages

%%%%%%%%%%%%%%%%%%%%%%%%%%%%%%%%%%%%%%%%%%%%%%%%%
%%% Title page                                %%%
%%%%%%%%%%%%%%%%%%%%%%%%%%%%%%%%%%%%%%%%%%%%%%%%%
\period{Fall 2023}
\dasatype{Master Thesis}
\students{Nicolas N. Koch}
\mainreaderprefix{Advisor:}
\mainreader{Prof.\ Dr.\ Peter L. Bühlmann}
\alternatereaderprefix{Co-Advisors:}
\alternatereader{}
\submissiondate{2 April 2024}
\title{On-Line Comparison of Volatility Models \\ via Sequential Monte Carlo}

\maketitle %- Titelseite wird abgeschlossen
\cleardoublepage
 %%~~~~~~~~~~~~~~~~~~~~~~~~~~~~~~~~~~~~~~~~

%%%%%%%%%%%%%%%%%%%%%%%%%%%%%%%%%%%%%%%%%%%%%%%%%
%%% Insert here acknowledgements and abstract %%%
%%%%%%%%%%%%%%%%%%%%%%%%%%%%%%%%%%%%%%%%%%%%%%%%%
%% Dedication (optional)
\markright{}
\vspace*{\stretch{1}}
\begin{center}
    ...and of course shortly after kindergarten, \\ you learned how to solve such an equation. \\[6pt]
    \emph{-- Josef Teichmann}
\end{center}
\vspace*{\stretch{2}}

\vspace*{\stretch{1}}
\begin{center}
    Equations are often easy to write down, \\ but rarely easy to compute. \\[6pt]
    \emph{-- Josef Teichmann}
\end{center}
\vspace*{\stretch{2}}

% Preface (optional)
\newpage
\markboth{Preface}{Preface}
\chapter*{Preface}

First words and acknowledgements.



% Abstract should not be longer than one page.
\newpage
\markboth{Abstract}{Abstract}
\chapter*{Abstract}

Short summary of your thesis.

%%%%%%%%%%%%%%%%%%%%%%%%%%%%%%%%%%%%%%%%%%%%%%%%%
%%% Table of contents and list of figures and %%%   
%%% tables (no need to change this usually)   %%%
%%%%%%%%%%%%%%%%%%%%%%%%%%%%%%%%%%%%%%%%%%%%%%%%%
\newpage
\tableofcontents
\newpage
\listoffigures
\newpage
\listoftables

%% Notations and glossary (optional)
\cleardoublepage
%\phantomsection
\addcontentsline{toc}{chapter}{\protect\numberline{}{Notation}}
\markboth{Notation}{Notation}
\chapter*{Notation}
\label{c:Notation}

\begin{itemize}
	\item $X$: random variable
	\item $x$: realization of random variable $X$
	\item $\sigma$: volatility of deterministic model
	\item $V$: volatility of stochastic model
\end{itemize}

\cleardoublepage
\pagenumbering{arabic}%--- switch back to standard numbering 


%%%%%%%%%%%%%%%%%%%%%%%%%%%%%%%%%%%%%%%%%%%%%%%%%
%%% Your text... Either write here directly,  %%%
%%% or even better: write in separate files   %%%
%%% that you just have to include here.       %%% 
%%%%%%%%%%%%%%%%%%%%%%%%%%%%%%%%%%%%%%%%%%%%%%%%%
\chapter{Introduction} 

Volatility is key quantity for pricing options, creating hedges, quantitative trading strategies, ...

Non-Bayesian model selection may not generalize well to new datasets

Online estimation estimation allows incorporating new data after fitting model without re-fitting, constant memory/computation cost. Valuable in finance, where data arrives continuously. Offline estimation results are conditional on length of time series; not clear how best model dependent on sample size
\chapter{Volatility Models}
\label{sec:models}

A volatility model is a model for the conditional standard deviation of the next value of a path given the past. Formally, we consider discrete $\R$-valued price processes with non-negative time indices, $S: \N_0 \to \R$, adapted to the filtration $\mathcal{F}_t = \sigma(\{S_{t'}, t' \leq t\})$. Based on the price process, define the \emph{log-returns} $(X_t)$ such that $X_t := \Delta \log S_{t-1} = \log \frac{S_t}{S_{t-1}}$. Conventionally, volatility models are constructed to model the latter.
inherently mean-zero, make no attempt at modeling expectation


\section{Deterministic Volatility Models}

In deterministic volatility models, the current volatility level is known given the filtration, the model, and the values of its parameters.
\begin{align*}
X_t ~ &= ~ \sigma_t Z_t \\
\sigma_t^2 ~ &= ~ f_\theta(t, S_{0:t-1})
\end{align*}

\begin{lemma}[Ito's Lemma]
Consider a stochastic process $(S_t)$ that satisfies the stochastic differential equation $\d S_t = \mu_t \d t + \sigma_t \d W_t$. Let $f(t, S_t)$ be a twice differentiable function mapping to $\R$. Then,
\begin{equation*}
\d f ~ = ~ \left( \partial_t f + \mu_t \partial_S f + \frac{1}{2} \sigma_t^2 \partial^2_S f \right) \d t + \sigma_t \partial_S f \d W_t
\end{equation*}
\end{lemma}

\begin{equation*}
\d \log S_t ~ = ~ \frac{\sigma_t}{S_t} \d \widetilde{W}_t
\end{equation*}

GBM
\begin{equation*}
\d S_t ~ = ~ \mu S_t \d t + \sigma S_t \d W_t
\end{equation*}
\begin{equation*}
X_t ~ = ~ \sigma Z_t
\end{equation*}

Ornstein-Uhlenbeck
\begin{equation*}
\d S_t ~ = ~ \kappa(\mu-S_t) \d t + \sigma \d W_t
\end{equation*}
\begin{equation*}
X_t ~ = ~ \frac{\sigma}{S_{t-1}} Z_t
\end{equation*}

GARCH
\begin{align*}
X_t ~ &= ~ \sigma_t Z_t \\
\sigma_t^2 ~ &= ~ \omega + \alpha X_{t-1}^2 + \beta \sigma_{t-1}^2
\end{align*}

\begin{equation*}
\sigma_t^2 ~ = ~ \omega \sum_{j=1}^t \beta^{j-1} + \alpha \sum_{j=1}^t \beta^{j-1} X_{t-j}^2 + \beta^{t-1} \sigma_1^2
\end{equation*}

E-GARCH
\begin{align*}
X_t ~ &= ~ \sigma_t Z_t \\
\log \sigma_t^2 ~ &= ~ \omega + \alpha \Big( |Z_{t-1}| - \E[|Z_{t-1}|] \Big) + \gamma \widetilde{Z}_{t-1} + \beta \log \sigma_t^2
\end{align*}

GJR-GARCH
\begin{align*}
X_t ~ &= ~ \sigma_t Z_t \\
\sigma_t^2 ~ &= ~ \omega + \Big( \alpha + \gamma 1(X_{t-1} > 0) \Big) X_{t-1}^2 + \beta \sigma_{t-1}^2
\end{align*}



\section{Stochastic Volatility Models}

Stochastic volatility models assume that volatility is a separate latent stochastic process.
\begin{align*}
X_t ~ &= ~ V_t Z_t \\
V_t ~ &= ~ f_\theta(t, V_{1:t-1}, \widetilde{Z}_{1:t})
\end{align*}
where $(\widetilde{Z}_t)$ is another standard white noise innovation process that may be correlated with $(Z_t)$, $\Cov(Z_t, \widetilde{Z}_t) = \rho$.

Stochastic volatility
\begin{align*}
X_t ~ &= ~ V_t Z_t \\
\log V_t^2 ~ &= ~ \omega(1-\alpha)  + \alpha \log V_{t-1}^2 + \xi \widetilde{Z}_t
\end{align*}

Heston model
\begin{align*}
\d S_t ~ &= ~ \mu S_t \d t + V_t S_t \d W_t \\
\d V_t^2 ~ &= ~ \kappa(\nu - V_t^2) \d t + \xi {V_t} \d \widetilde{W}_t
\end{align*}

\begin{align*}
X_t ~ &= ~ V_t Z_t \\
\log V_t ~ &= ~ \log V_{t-1}^2 + \kappa \left( \frac{\nu}{V_{t-1}^2} - 1 \right) - \frac{1}{2} \frac{\xi^2}{V_{t-1}^2} + \frac{\xi}{V_{t-1}} \widetilde{Z}_t
\end{align*}
$\Cov(Z_t, \widetilde{Z}_t) = \rho$

SABR model
\begin{align*}
\d S_t ~ &= ~ V_t S_t^\beta \d W_t \\
\d V_t^2 ~ &= ~ \alpha V_t^2 \d \widetilde{W}_t
\end{align*}

\begin{align*}
X_t ~ &= ~ V_t S_{t-1}^{\beta-1} Z_t \\
\log V_t^2 ~ &= ~ \alpha \widetilde{Z}_t
\end{align*}

Quintic Ornstein-Uhlenbeck
\begin{align*}
\d S_t ~ &= ~ V_t S_t \d W_t \\
\d V_t^2 ~ &= ~ \sqrt{\xi} \frac{p(X_t)}{\sqrt{\E[p(X_t)^2]}}, \quad p(x) = \alpha_0 + \alpha_1 x + \alpha_3 x^3 + \alpha_5 x^5 \\
\d X_t ~ &= ~ \varepsilon^{-1} (H - 1/2) X_t \d t + \varepsilon^{H-1/2} \d W_t
\end{align*}


\section{Neural Volatility Models}

Neural GARCH
\begin{align*}
X_t ~ &= ~ \sigma_t Z_t \\
\sigma_t^2 ~ &= ~ f_\phi(X_{t-1}, \sigma_{t-1}^2)
\end{align*}

Neural stochastic volatility
\begin{align*}
X_t ~  &= ~ V_t Z_t \\
V_t^2 ~ &= ~ f_\phi(V_{t-1}, \widetilde{Z}_t)
\end{align*}




\section{Local \& Path-Dependent Volatility}

The models introduced thus far all make one strong assumption: that the volatility today is uniquely determined by metrics of yesterday. In general, volatility may depend on the entire past path of observed as well as latent processes.

Consider first deterministic volatility models.

One approach is to handcraft a specific form in which path-dependence enters the model. For instance, \cite{Guyon22} propose a model wherein volatility is a simple affine function of a measure of the recent trend $\mathcal{T}$ and recent volatility $\Sigma$,
\begin{align*}
\sigma_t ~ &= ~ \beta_0 + \beta_1 \mathcal{T}_t + \beta_2 \Sigma_t
\end{align*}
where they define
\begin{align*}
\mathcal{T}_t ~ :=& ~ \sum_{t' \leq t} k_1(t-t') X_{t'} \\
\Sigma_t^2 ~ :=& ~ \sum_{t' \leq t} k_2(t-t') X_{t'}^2
\end{align*}
for some kernel functions $k_1, k_2: \R_+ \to \R_+$.

As in the previous section, neural networks allow to take a more data-driven approach also for path-dependent settings. \emph{Recurrent neural networks} (RNNs) maintain a hidden state $\pmb{h}_t$ which is recursively updated over time and hence are theoretically able to capture all relevant information of a path of observations.

A simple application of RNNs to deterministic volatility models might look as follows:
\begin{align*}
X_t ~ &= ~ \sigma_t Z_t \\
\sigma_t^2 ~ &= ~ f_\phi(\pmb{h}_t) \\
\pmb{h}_t ~ &= ~ g_\phi(\pmb{h}_{t-1}, X_{t-1})
\end{align*}

Similar applications have been proposed for stochastic volatility, e.g. by \citet{Luo18}.

Although RNNs are theoretically able to learn arbitrary functions on path spaces, they are limited by numerical issues in practice. Since the hidden state update $g_\phi(\cdot)$ is applied recursively, gradient information tends to be lost over time ("vanishing gradient problem"), and hence long-range dependencies are only very inefficiently learned. The more refined GRU or LSTM architectures are thus often preferred in practice. However, the idea of path-dependence is the same, and as is more illustrated in detail in Sect. \ref{sec:RC}, we will only take alternative approaches which merely approximately behave like neural models, anyway.

Denote by $(\widehat{X}_t) = ((t, X_t))$ the time-extended process of the log-returns.
\begin{align*}
X_t ~ &= ~ \sigma_t Z_t \\
\sigma_t^2 ~ &= ~ \pmb{\beta}^\top \Sig(\widehat{X}_{1:t-1}) + \beta_0
\end{align*}

\begin{align*}
X_t ~&= ~ V_t Z_t \\
V_t^2 ~ &= ~ \pmb{\beta}^\top \E \left[ \Sig(V_{1:t-1}) \mid X_{1:t-1} \right] + \beta_0
\end{align*}



\begin{table}[]
\caption{Models: Overview}
\begin{tabular}{|l|l|}
\hline
\textbf{Deterministic} & \textbf{Stochastic} \\ \hline
GBM                    & StochVol            \\ \hline
Ornstein-Uhlenbeck     & Heston              \\ \hline
GARCH                  & SABR                \\ \hline
E-GARCH                & Neural              \\ \hline
GJR-GARCH              & Signature           \\ \hline
Neural                 &                     \\ \hline
Signature              &                     \\ \hline
\end{tabular}
\end{table}



\chapter{Bayesian Model Assessment, Selection, \& Comparison}

\section{Basics of Bayesian Statistics}

\begin{equation*}
\pi(\theta \mid x) ~ = ~ \frac{ \pi(\theta) \, p(x \mid \theta) }{ p(x) }
\end{equation*}

Bernstein-von Mises theorem


\section{Model Assessment}

Goals: (i) identify "correct" model, (ii) identify model with superior predictive accuracy

in-sample predictive error: downward biased, even for Bayesian models; cross-validation popular in practice, but computationally expensive; information criteria adjust for bias

Bayesian cross-validation: same as standard cross-validation, but assess mean predictive density on test set $p(X_{test}) = \int p(X_{test} \mid \theta) \d \pi(\theta \mid X_{train})$

asymptotically (for $N \to \infty$) xyIC equivalent to cross-validation (also for time series??)

\begin{definition}[Evidence]
Consider a probabilistic, parametric model for the data-generating process of a dataset $X_{1:t}$ parametrized by $\theta \in \Theta$ with prior law $\pi_0(\theta)$. The evidence $Z_t$ of the model given data $X_{1:t}$ is
\begin{equation*}
Z_t ~ := ~ \int_\Theta p(X_{1:t} \mid \theta) \d \pi_0(\theta)
\end{equation*}
\end{definition}
Equivalently, the evidence of a model can also be seen as the marginal likelihood $p(X_{1:t})$ of the data or as the expected likelihood of the dataset, $\E_{\pi_0} \Big[ p(X_{1:t} \mid \theta) \Big]$.

\subsection{Information Criteria}

\begin{definition}[Bayesian Information Criterion]
\end{definition}



\begin{definition}[Deviance Information Criterion]
\begin{align*}
\textrm{DIC} ~ :=& ~ -2 \log p(X \mid \hat{\theta}) + 2 k_{\textrm{eff}} \\
k_{\textrm{eff}} ~ =& ~ 2 \left( \log p(X \mid \hat{\theta}) - \E_{\pi_t} \left[ \log p(X \mid \theta) \right] \right)
\end{align*}
\end{definition}

\begin{definition}[Watanabe--Akaike Information Criterion]
\end{definition}

plug-in predictive distribution
\begin{equation*}
p(X_{t+1} \mid X_{1:t}) ~ = ~ \int_\Theta p(X_{t+1} \mid X_{1:t}, \theta) \d \pi_t(\theta \mid X_{1:t})
\end{equation*}

\section{Model Comparison}

\subsection{Bayes Factors}
\chapter{Sequential Monte Carlo Methods} 

If $\mathcal{V}$ was finite, the posterior is similarly tractable and can be computed following a procedure known as the Baum-Welch algorithm.

\section{Basics of Monte Carlo Estimation}

Consider a random variable $\theta \in \Theta$ distributed according to (cumulative) distribution function $\pi$. On the one hand, we are interested in the expectation of a test function $f: \Theta \to \R^d$,
\begin{align*}
\E_\pi[f(\theta)] ~ = ~ \int_\Theta f(\theta) \d \pi(\theta)
\end{align*}
Depending on $\Theta$, $f$, and $\pi$, this integral quickly becomes impossible to solve. The central dogma of Monte Carlo estimation is to exploit a approximation of $\pi$ based on individual samples, or "particles" therefrom, giving rise to its particle approximation, or empirical distribution function
\begin{equation*}
\hat{\pi}(\theta) ~ = ~ \frac{1}{n} \sum_{i=1}^n 1(\theta \leq \theta_i)
\end{equation*}
Since $\hat{\pi}$ is piecewise constant and always jumps by $\frac{1}{n}$, such a particle approximation of $\pi$ reduces the problem to a much more tractable summation task, namely
\begin{equation*}
\E_{\hat{\pi}}[f(\theta)] ~ = ~ \int_\Theta f(\theta) \d \hat{\pi}(\theta) ~ = ~ \frac{1}{n} \sum_{i=1}^n f(\theta_i)
\end{equation*}
This basic idea can be applied in any setting where distributions can be approximated by particle approximations and test functions can be evaluated pointwise. For instance, under a Bayesian paradigm, we are interested in the \emph{evidence} of a model given some data $x$, which is defined as
\begin{equation*}
p(x) ~ = ~ \E_\pi[p(x \mid \theta)] ~ = ~ \int_\Theta p(x \mid \theta) \d \pi(\theta)
\end{equation*}

\begin{equation*}
\widehat{p}(x) ~ = ~ \frac{1}{n} \sum_{i=1}^n p(x \mid \theta_i)
\end{equation*}
i.e. the average likelihood of the data over the sampled parameters.


\section{Importance Sampling}

Consider the following trivial identities:
\begin{equation*}
\pi(\theta) ~ = ~ \int_{\theta' \leq \theta} \d \pi(\theta') 
	= ~ \int_{\theta' \leq \theta} \frac{\d \pi(\theta')}{\d \nu(\theta')} \d \nu(\theta') ~ = ~ \int_{\theta' \leq \theta} w(\theta') \d \nu(\theta')
\end{equation*}
and hence for $x_i \simiid q$, $i=1,...,n$,
\begin{equation*}
\hat{\pi}(\theta) ~ = ~ \frac{1}{N} \sum_{i=1}^N w(\theta_i) 1(\theta \leq \theta_i)
\end{equation*}
where $\nu$ is another distribution function called the "proposal" and $w(\theta) := \frac{\d \pi(\theta)}{\d \nu(\theta)}$ is the Radon-Nikodym derivative of $\pi$ w.r.t. $\nu$. This suggests that if it is not possible to sample from $\pi$, but from $\nu$ and to evaluate $w(\theta)$ pointwise, we can still use Monte Carlo techniques to estimate quantities of $\pi$.

The weight $w(\theta)$ corrects for the fact that $\theta$ was drawn from the wrong distribution. Intuitively, if a specific particle $\theta_i$ is common under the target $\pi$ but rare under the proposal $\nu$, it needs to be given high weight, and vice versa.

Formally, the only requirement is that $w$ is always well-defined, i.e. that $\sup_\theta \frac{\d \pi(\theta)}{\d \nu(\theta)} \leq M$ for some $M < \infty$. This in turn requires that $\d \nu(\theta) = 0$ only when $\d \pi(\theta) = 0$, implying that the support of $\nu$ must contain the support of $\pi$. Additionally, $\nu$ must have fatter tails than $\pi$ (if the latter's support is unbounded). Otherwise, if e.g. the density of $\nu$ decayed more rapidly towards zero than that of $\pi$, $w$ would of course grow indefinitely.

In many practical applications, $\pi$ is only known up to a multiplicative factor. That is, we know $\gamma(\theta)$ such that
\begin{equation*}
\frac{1}{Z} \gamma(\theta) ~ = ~ \pi(\theta)
\end{equation*}
where $Z := \int_\Theta \d \gamma(\theta)$ is an unknown normalizing constant ensuring that $\pi(\theta) \tendsto{\theta \to \infty} 1$. In this case, one has to work with the unnormalized weights $W(\theta) := \frac{\d \gamma(\theta)}{\d \nu(\theta)}$. It turns out that these are unbiased estimates of the unknown normalizing constant $Z$:
\begin{align*}
\E_\nu \Big[ W(\theta) \Big] ~ &= ~ \int_\Theta W(\theta) \d \hat{\pi}(\theta) ~ = ~ \int_\Theta \d \gamma(\theta) ~ =: ~ Z
\end{align*}
suggesting use of the estimator
\begin{equation*}
\widehat{Z} ~ := ~ \frac{1}{N} \sum_{i=1}^N W(\theta_i)
\end{equation*}
In the context of model selection, $Z$ is itself of prime interest, and hence this is a relevant result.
\begin{equation*}
\V_\nu(\widehat{Z}) ~ = ~ \frac{1}{N} \V_\nu \big( W(\theta) \big)
\end{equation*}
Since, if $\nu = \pi$, the unnormalized weights are always $1$, and hence $\widehat{Z}$ has zero variance, it is clear that the optimal proposal should be as close to the target as possible.
\begin{equation*}
\nu^* ~ = ~ \pi
\end{equation*}

\begin{align*}
\hat{\pi}(\theta) ~ &= ~ \frac{1}{n} \sum_{i=1}^N \frac{W(\theta_i)}{\sum_{i=1}^n W(\theta_i)} 1(\theta \leq \theta_i)
\end{align*}

\begin{equation*}
\ESS ~ := ~ \frac{N}{1 + \V_\nu(W(\theta))}
\end{equation*}


\section{Sequential Importance Sampling}

Suppose our unknown quantity $\theta$ is a parameter in the distribution $p(x)$ of $x$. Observing data $x_{1:t} = \{x_1,...,x_t\}$ carrying information about this quantity gives rise to the target distribution $\pi_t(\theta)$.

If all the observations $x_{1:t}$ arrive simultaneously, batch inference is a popular approach. Therein, one attempts to estimate the distribution by computing resp. approximating $\pi_t$ directly. Some of the most popular batch inference methods include Markov chain Monte Carlo (MCMC) and Hamiltonian Monte Carlo (HMC).

When a new observation $x_{t+1}$ arrives, however, batch inference methods have no direct way to use $\pi_t$. Consequently, when data arrives sequentially, one is left with no other choice than running the entire analysis again every time. This is a severe limitation in fields including finance, where often data arrives and hence models have to be updated daily if not more frequently.

In sequential Bayesian inference, the idea is to construct intermediate distributions and sequentially update them to incorporate new observations. That is, they are inherently on-line by construction. This is of primary interest when a model needs to be estimated today but new data will inevitably arrive tomorrow. It has also been demonstrated to have attractive properties in batch inference.

\begin{equation*}
X_t \mid (X_{t-1} = x_{t-1}, \theta) ~ \sim ~ P_\theta(\cdot \mid x_{t-1})
\end{equation*}
where $P_\theta(\cdot) = P(\cdot \mid \theta)$.
\begin{equation*}
\pi_t(\theta) ~ := ~ \pi(\theta \mid x_{1:t})
\end{equation*}

\begin{align*}
\pi_t(\theta) ~ &= ~ \frac{1}{Z_t} \gamma_t(\theta); \quad Z_t ~ = ~ \int_\Theta \d \gamma_t(\theta) \\
\d \gamma_t(\theta) ~ &= ~ P_\theta(x_{1:t}) \d \pi_0(\theta) ~ = ~ \d \pi_0(\theta) \cdot \prod_{k=1}^t P_\theta(x_k \mid x_{k-1}) \\
\end{align*}
Through another trivial identity, we obtain a recursive expression for the weights, namely
\begin{align*}
W_t(\theta) ~ := ~ \frac{\d \gamma_t(\theta)}{\d \nu_t(\theta)} ~ &= ~ \frac{\d \gamma_{t-1}(\theta)}{\d \nu_{t-1}(\theta)} \cdot \frac{\d \gamma_t(\theta) \, / \d \gamma_{t-1}(\theta)}{\d \nu_t(\theta) \, / \d \nu_{t-1}(\theta)} \\
	&= ~ W_{t-1}(\theta) \cdot \alpha_t(\theta)
\end{align*}
where $\alpha_t(\theta) := \frac{\d \gamma_t(\theta) \, / \d \gamma_{t-1}(\theta)}{\d \nu_t(\theta) \, / \d \nu_{t-1}(\theta)}$ is the weight update from $t-1$ to $t$. If use a constant proposal distribution at each iteration, ignoring new information from the observations $x_{1:t}$, then $\alpha_t$ corresponds to the conditional likelihood of the $t^{\textrm{th}}$ observation
\begin{equation*}
\frac{\d \gamma_t(\theta)}{\d \gamma_{t-1}(\theta)} ~ = ~ P_\theta(x_t \mid x_{t-1})
\end{equation*}
An estimate for the normalizing constant is obtained akin to standard importance sampling
\begin{equation*}
\widehat{Z}_t ~ = ~ \frac{1}{n} \sum_{i=1}^N W_t(\theta_i)
\end{equation*}

\section{Resampling}

weight degeneracy

multinomial, stratified, systematic

\begin{algorithm}
\label{alg:IBIS}
\caption{Iterated Batch Importance Sampling (IBIS)}
\hspace*{\algorithmicindent} \textbf{Input} $\pi_0$, $P$, $K_t$, \texttt{resample} \\
\hspace*{\algorithmicindent} \textbf{Output} $\widehat{\pi}_t, \widehat{Z}_{1:t}$
\begin{algorithmic}
\State $\theta_0^{(i)} \sim \pi_0(\theta)$ \algorithmiccomment{generate particles from prior}
\State $W_0^{(i)} \setto 1$ \algorithmiccomment{initialize uniform weights}
\For{$t=1$ to $T$}
	\If{$\ESS < \ESSmin$}
		\State $\tilde{\theta}_{t-1}^{(i)} \setto \resample(\theta_{t-1}^{(1:N)}; w_{t-1}^{(1:N)})$ \algorithmiccomment{resample particles}
		\State $\widetilde{W}_{t-1}^{(i)} \setto 1$ \algorithmiccomment{reset weights}
	\Else
		\State $\tilde{\theta}_{t-1}^{(i)} \setto \theta_{t-1}^{(i)}$ \algorithmiccomment{maintain particles}
		\State $\widetilde{W}_{t-1}^{(i)} \setto W_{t-1}^{(i)}$ \algorithmiccomment{maintain weights}
	\EndIf
\State $\theta_t^{(i)} \sim K_t(\cdot \mid \widetilde{\theta}_{t-1}^{(i)})$ \algorithmiccomment{propagate particles}
\State $\alpha_t^{(i)} \setto P(x_t \mid x_{t-1}; \theta_t^{(i)})$ \algorithmiccomment{compute weight update}
\State $W_t^{(i)} \setto \widetilde{W}_{t-1}^{(i)} \cdot \alpha_t^{(i)}$ \algorithmiccomment{update weights}
\State $w_t^{(i)} \setto W_t^{(i)} / \sum_{j=1}^N W_t^{(j)}$ \algorithmiccomment{re-normalize weights}
\EndFor
\end{algorithmic}
\end{algorithm}

In the original paper, \citeauthor{Chopin02} \citeyearpar{Chopin02} proposes to use an adaptive MCMC scheme, wherein the target is approximated through a Gaussian distribution with the empirical mean $\hat{\mu}$ and covariance $\widehat{\Sigma}$ of the particles. New particles are then proposed via a Gaussian random walk, so that $K_t(\cdot \mid \theta) = \mathcal{N}(\theta, \widehat{\Sigma})$


\section{Particle Filtering}

In stochastic models, the evolution of $(X_t)$ depends that of another, latent stochastic process $(V_t)$. The models proposed in Sect. \ref{sec:models} are all of the form
\begin{align*}
X_t \mid (X_{t-1}=x_{t-1}, V_t=v_t) ~ &\sim ~ P(\cdot \mid x_{t-1}, v_t) \\
V_t \mid V_{t-1}=v_{t-1} ~ &\sim ~ K(\cdot \mid v_{t-1})	
\end{align*}
where $P$ is a likelihood function for the observations and $K$ is a transition kernel. In stochastic volatility and other models, $V_t$ can be interpreted as the current "state" of the system, hence such models are referred to as \emph{state-space models}. For the moment, suppose that $P$ and $K$ do not depend on any (unknown) parameters.

Our target distribution is that of the latent path $v_{1:t}$,
\begin{equation*}
\pi_t(v_{1:t}) ~ = ~ \frac{1}{Z_t} \prod_{j=1}^t K_\theta(v_j \mid v_{j-1}) P_\theta(x_j \mid x_{j-1}, v_j) ~ = ~ \frac{1}{Z_t} \gamma_t(v_{1:t})
\end{equation*}
where we define prior distributions $P(x_1 \mid x_0, v_1) := \eta_X(x_1 \mid v_1)$ and $K(v_1 \mid v_0) := \eta_V(v_1)$. From this distribution on the path space, we can deduce all distributions of interest through integration. These include the \emph{filtering distribution} $\pi_t(v_t \mid v_{1:t-1})$, evidence $p(x_{1:t}) = \int \pi_t(v_t \int)$

We can again readily extend the sequential importance sampling framework to approximate the new target distribution. Borrowing the expression from the static setting, we have the recursion for the weights
\begin{equation*}
W_t(v_{1:t}) ~ = ~ W_{t-1}(v_{1:t}) \cdot \frac{\d \gamma_t(v_t \mid v_{1:t-1})}{\d \nu_t(v_t \mid v_{1:t-1})}
\end{equation*}
where the numerator of the weight update is again just a conditional likelihood, this time of the new latent quantity $v_t$ given the previous ones, $v_{1:t-1}$, and the observations $x_{1:t}$. It is given by
\begin{equation*}
\d \gamma_t(v_t \mid v_{1:t-1}) ~ = ~ P(x_t \mid v_t) K(v_{1-t} \mid v_t)
\end{equation*}
This gives rise to the so called \emph{guided particle filter} outlined in Alg. \ref{alg:GPF}.

\begin{algorithm}
\label{alg:GPF}
\caption{(Guided) Particle Filter}
\hspace*{\algorithmicindent} \textbf{Input} $\eta_V$, $\eta_X$, $P$, $K$, $\ESSmin$, \texttt{resample}
\begin{algorithmic}
\State $v_1^{(i)} \sim \eta_V(v_1)$ \algorithmiccomment{generate particles from prior}
\State $W_1^{(i)} \setto 1$ \algorithmiccomment{initialize weights uniformly}
\State $w_1^{(i)} \setto N^{-1}$
\For{$t=2$ to $T$}
	\State $\ESS = 1 / \sum_{j=1}^N (W_t^{(i)})^2$ \algorithmiccomment{compute ESS}
	\If{$\ESS < \ESSmin$}
		\State $\widetilde{v}_{t-1}^{(i)} \sim \resample(v_{t-1}^{(1:N)}; w_{t-1}^{(1:N)})$ \algorithmiccomment{resample particles}
		\State $\widetilde{W}_{t-1}^{(i)} \setto 1$ \algorithmiccomment{reset weights}
	\Else
		\State $\widetilde{v}_{t-1}^{(i)} \setto v_{t-1}^{(i)}$ \algorithmiccomment{maintain particles}
		\State $\widetilde{W}_{t-1}^{(i)} \setto W_{t-1}^{(i)}$ \algorithmiccomment{maintain weights}
	\EndIf
\State $v_t^{(i)} \sim K(\cdot \mid \widetilde{v}_{t-1}^{(i)})$ \algorithmiccomment{propagate particles}
\State $\alpha_t^{(i)} \setto K(v_t^{(i)} \mid v_{t-1}^{(i)}) P(x_t \mid x_{t-1}, v_t^{(i)}) \, / \d \nu_t(v_t \mid v_{t-1})$ \algorithmiccomment{compute weight update}
\State $W_t^{(i)} \setto \widetilde{W}_{t-1}^{(i)} \cdot \alpha_t^{(i)}$ \algorithmiccomment{update weights}
\State $w_t^{(i)} \setto W_t^{(i)} / \sum_{j=1}^N W_t^{(j)}$ \algorithmiccomment{re-normalize weights}
\EndFor
\end{algorithmic}
\end{algorithm}

In particle filtering, unlike in IBIS, particles do not necessarily have to be moved after resampling to introduce variation, since they are propagated either way.

A problem that remains is that of the choice of proposal distribution. Akin to standard importance sampling, the optimal proposal is again the true law,
\begin{equation*}
\d \nu_t^*(v_t \mid v_{t-1}) ~ = ~ \pi_t(v_t \mid v_{t-1}) ~ = ~ \frac{P(x_t \mid v_t) K(v_t \mid v_{t-1})}{\int P(x_t \mid v_t) K(v_t \mid v_{t-1}) \d x_t}
\end{equation*}
However, this is often intractable in practice. A popular alternative is to ignore the information about $v_t$ brought by the $x_t$ and simply use $\nu_t(v_t \mid v_{t-1}) = K(v_t \mid v_{t-1})$. This is referred to as the \emph{bootstrap proposal}. Naturally, its inefficiency is smaller whenever $x_t$ not very informative about $v_t$.


\section{SMC Samplers}

\begin{align*}
\pi_t(v_{1:t}, \theta) ~ &= ~ \frac{1}{Z_t} \gamma_t(v_{1:t}, \theta) \\
\d \gamma_t(v_{1:t}, \theta) ~ &= ~ L(x_{1:t} \mid v_{1:t}, \theta) \d \pi_0(v_{1:t}, \theta) \\
	&= ~ L(x_1 \mid v_1, \theta) \d \pi_0(v_1, \theta) \prod_{k=2}^t L(x_k \mid v_k, \theta) K(v_k \mid v_{k-1}, \theta) \\
Z_t ~ &= ~ \int L(x_{1:t} \mid v_{1:t}, \theta) \d \pi_0(v_{1:t}, \theta)
\end{align*}

\begin{align*}
W_t(v_{1:t}) ~ = ~ \frac{\d \gamma_t(v_{1:t})}{\d \nu_t(v_{1:t})}
\end{align*}

\begin{align*}
W_t(v_{1:t}, \theta) ~ := ~ \frac{\d \gamma_t(v_{1:t}, \theta)}{\d \nu_t(v_{1:t}, \theta)} ~ &= ~ \frac{\d \gamma_{t-1}(v_{1:t-1}, \theta)}{\d \nu_{t-1}(v_{1:t-1}, \theta)} \cdot \frac{\d \gamma_t(v_t \mid v_{1:t-1}, \theta)}{\d \nu_t(v_t \mid v_{1:t-1}, \theta)} \\
	&= ~ W_{t-1} \cdot \frac{\d \gamma_t(v_t \mid v_{t-1}) \d \gamma_t(x_t \mid x_{t-1}, v_t)}{\d \nu_t(v_t \mid v_{t-1}) \d \nu_t(x_t \mid x_{t-1}, v_t)}
\end{align*}

\subsection{Rao-Blackwell Particle Filtering}

\begin{theorem}[Rao-Blackwell]
\begin{equation*}
\E \left[ \left( \E \big[\hat{\theta}(X_{1:t}, \tau) \mid \tau(X_{1:t}) \big] - \theta \right)^2 \right] ~ \leq ~ \E \left[ \left( \hat{\theta}(X_{1:t}) - \theta \right)^2 \right]
\end{equation*}
\end{theorem}

\begin{lemma}[Student-t distribution]
If $\tau \sim \textrm{Gamma}(\nu/2, \nu/2)$ and $X \,|\, \tau ~ \sim ~ \mathcal{N}(0, \tau^{-1})$, then $X \sim t_\nu$.
\end{lemma}



\chapter{Reservoir Computing} 
\label{sec:RC}

\section{Random Neural Networks}

\emph{random feature models},

\emph{Extreme Learning Machines} (ELMs). An ELM is a shallow feedforward neural network with randomly drawn inner weights. In the one-dimensional case, we have
\begin{align*}
\textrm{ELM}(\pmb{x}) ~ :=& ~ \pmb{\beta}^\top \phi(\pmb{Ax} + \pmb{b}) + \beta_0 \\
\pmb{A}_{ij}, \pmb{b}_i ~ \simiid & ~ 
\end{align*}

\emph{echo state network}




\section{Signature Methods}

\begin{definition}[Signature]
Let $\pmb{u} \in BV([0,T]; \R^p)$. The signature component associated with the word $w \in \{0, ..., d\}^k$ of length $k \geq 0$ is
\begin{equation*}
\Sig^{(w)}(\pmb{u}) ~ := ~ \int_{0 \leq t_1 \leq \cdots \leq t_k \leq t} \d u^{(i_1)}_{t_1} \cdots \d u^{(i_k)}_{t_k}
\end{equation*}
\end{definition}

Signature of time-extended path is point-separating. 

Continuous path functionals can be uniformly approximated on compact sets by linear functionals of the time-extended signature evaluated at the final time

\begin{theorem}[\citep{Cuchiero22}]
Let $\pmb{u}$ be a càdlàg path and $f$ a continuous function on path space. Then, for every $\varepsilon > 0$, there exists a linear functional $\ell$ such that
\begin{equation*}
\sup_{\pmb{u} \in K} \big|f(\pmb{u}) - \ell(\Sig(\pmb{u})) \big| ~ \leq ~ \varepsilon
\end{equation*}
\end{theorem}

Hence signature effective feature extraction method for time series and well suited to capture path-dependency.
\begin{align*}
X_t ~ &= ~ \sigma_t Z_t \\
\sigma_t ~ &= ~ \beta_0 + \pmb{\beta}^\top \Sig(X_{1:t})
\end{align*}

Truncated signature
\begin{equation*}
\Sig^{M}(\pmb{u}) ~ := ~ \int_{0 \leq t_1 \leq \cdots \leq t_k \leq t} \d u^{i_1}_{t_1} \cdots \d u^{i_k}_{t_k}
\end{equation*}

Randomized signature
\begin{theorem}[\citep{Cuchiero21}]
Consider random matrices $\pmb{A}_i \in \R^{q \times q}$, random vectors $\pmb{b}_i \in \R^q$, and activation function $\phi: \R \to \R$ (applied component-wise). The random vector $\widehat{\Sig} \in \R^q$ constructed as the solution to the controlled ODE
\begin{equation*}
\d \widehat{\Sig}_t ~ = ~ \sum_{i=1}^d \phi \Big( \pmb{A}_i \widehat{\Sig}_t + \pmb{b}_i \Big) \d u_t^{(i)}
\end{equation*}
with random initial condition $\widehat{\Sig}_0$ approximately preserves the geometry of $\Sig$, in the sense that
\begin{equation*}
(1-\varepsilon) \norm{\Sig(\pmb{u}_t) - \Sig(\pmb{u}_t)}_2 ~ \leq ~ \norm{\RandSig(\pmb{u}) - \RandSig(\pmb{u}')}_2 ~ \leq ~ (1+\varepsilon) \norm{\Sig(\pmb{u}) - \Sig(\pmb{u})}_2
\end{equation*}
\end{theorem}

\include{Summary}

%%%%%%%%%%%%%%%%%%%%%%%%%%%%%%%%%%%%%%%%%%%%%%%%%
%%% Bibliography                              %%%
%%%%%%%%%%%%%%%%%%%%%%%%%%%%%%%%%%%%%%%%%%%%%%%%%
\addtocontents{toc}{\vspace{.5\baselineskip}}
\cleardoublepage
%\phantomsection
\addcontentsline{toc}{chapter}{\protect\numberline{}{Bibliography}}
\bibliography{References}
%% All books from our library (SfS) are already in a BiBTeX file
%% 'Assbib.bib' (included here as well), using
% \bibliography{References,Assbib}
% ---------------------------------- instead of the above



%%%%%%%%%%%%%%%%%%%%%%%%%%%%%%%%%%%%%%%%%%%%%%%%% 
%%% Appendices (if needed, e.g. for R code)   %%%
%%%%%%%%%%%%%%%%%%%%%%%%%%%%%%%%%%%%%%%%%%%%%%%%%
\addtocontents{toc}{\vspace{.5\baselineskip}}
\appendix
\chapter{Complementary information}
\label{app:complement}


\include{Appendix2}


%%%%%%%%%%%%%%%%%%%%%%%%%%%%%%%%%%%%%%%%%%%%%%%%%% 
%%% Declaration of originality (Do not remove!)%%%
%%%%%%%%%%%%%%%%%%%%%%%%%%%%%%%%%%%%%%%%%%%%%%%%%%
%% Instructions:
%% -------------
%% fill in the empty document confirmation-originality.pdf electronically
%% print it out and sign it
%% scan it in again and save the scan in this directory with name
%% confirmation-originality-scan.pdf 
%%
%% General info on plagiarism:
%% https://www.ethz.ch/students/en/studies/performance-assessments/plagiarism.html 
\cleardoublepage
%\includepdf[pages={-}, frame=true,scale=1]{confirmation-originality-scan.pdf}
\end{document}

%%% Local Variables:
%%% mode: latex
%%% TeX-master: "MasterThesisSfS"
%%% End:
